\documentclass[11pt,a4paper,oneside]{report}

\begin{document}

\section{Model Formulation 1}
For standard pair-wise Markov Network, we have 
\begin{equation}
  P_\Phi (y) = \frac{1}{Z} \prod_{i=1}^{N} \Phi_i(y_i) \prod_{(i,j)\in E} \Phi_{ij}(y_i,y_j)
  \end{equation}

Modeling $y_i^k$ as a continuous variable(higher value indicates higher confidence that segment i is of class k), we can define the log of the node potential as 
$ log \Phi_i(k) = (y_i^k -  w_n^k.x_i )^2$  and the log of the edge potential as $ log \Phi_{ij}(l,k) = w_e^{l,k}x_{ij} ( y_i^l -  y_j^k)^2$

Substituting these potential functions, we can write the log likelihood as: 

\begin{eqnarray*}
P_w (y|x) &=& \frac{1}{Z}exp(-\sum_{i=1}^{N} \sum_{k=1}^{K} (y_i^k - w^{k}_{n}.x_{i})^2 - \sum_{(i,j)\in E} \sum_{l,k=1}^{K} w_{e}^{l,k}.x_{ij}(y_i^l - y_j^k)^2), where\\
Z&=&\int_{-\infty}^{+\infty }{exp \bigg( - \sum_{i=1}^{N} \sum_{k=1}^{K} (y_i^k - w^{k}_{n}.x_{i})^2 - \sum_{(i,j)\in E} \sum_{l,k=1}^{K} w_{e}^{l,k}.x_{ij}(y_i^l - y_j^k)^2 \bigg) } d^{NK}y\\
&=&\int_{-\infty}^{+\infty }{exp \bigg(-y^T B y+a^T y\bigg) } d^{NK}y\\
&=&\frac{(\sqrt{\pi})^n}{\sqrt{det(B)}}exp(\frac{1}{4}y^T B^{-1} y)
\end{eqnarray*}

\subsection{Learning}
Maximum likelihood approach learning will require the evaluation of the partition function. We are trying to compute $Z_w$ analytically. If that is not possible, we can use the pseudo-likelihood estimate for approximate parameter learning. 



\section{Model Formulation 2}
Given labels $\{1,.., K\}$, $y_i^k$ is an indicator variable which is 1 if segment i has label k. 

For standard pair-wise Markov Network, we have 

\begin{equation}
  P_\Phi (y) = \frac{1}{Z} \prod_{i=1}^{N} \Phi_i(y_i) \prod_{(i,j)\in E} \Phi_{ij}(y_i,y_j)
  \end{equation}
  
  We can define the log of node potential as a linear function of the node features:  $log \Phi_i(k) = w_n^k.x_i$ , 
where $x_i$ is the feature vector of node i. Intuitively, the training process(margin-maximization) will try to pick $w_n^1...w_n^K$ such that $w_n^k.x_i>w_n^l.x_i \forall l$ if the correct label is k.\\
  Similarly, the log of edge potential can be defined as $log \Phi_{ij}(l,k) = w_{e}^{l,k}.x_{ij}$. The intuition is that $w_{e}^{l,k}$ will be high if segments of label k and label l appear frequently in configurations having relative features(relations) $x_{ij}$ . $x_{ij}$ could be a vector of indicator variables each indicating whether a particular relation holds between segments i and j.
However, instead that hard-coding the thresholds for relations like on-top, nearby, coplanar , we believe it would be better to include more informative continuous features in $x_{i,j}$ like difference of coordinates of centroids of the 2 segments, difference between mean normals etc.\\

Substituting these potential functions, we can write the log likelihood as: 

\begin{equation}
 \log P_w (y|x) = \sum_{i=1}^{N} \sum_{k=1}^{K} (w^{k}_{n}.x_{i})y_{i}^{k} + \sum_{(i,j)\in E} \sum_{l,k=1}^{K} (w_{e}^{l,k}.x_{ij})y_i^l y_j^k   -   \log Z_w(x)
\end{equation}

Doing some matrix manipulations(as shown in \cite{taskar2004learning}), we get:

\begin{equation}
 \log P_w (y|x) = wXy -logZ_w(x)
\end{equation}
where $w=[w_n w_e], y_n=[y_1^1...y_1^K....y_N^1...y_N^K],y_{i,j}^{l,k}=y_i^l*y_j^k,y_e=[ .....y_{i,j}^{1,1},y_{i,j}^{1,2}...y_{i,j}^{K,K} ....], y=[y_n,y_e]$ and X is constructed accordingly.
$w$ is of size $K*|x_i|+K^2*|x_{ij}|$

\section{Training}
The idea is to maximize the margin by which probability of correct label-sequence beats the probability of any incorrect label-sequence.

\begin{eqnarray*}
min: \frac{1}{2} ||w||^2 + C\epsilon\\
s.t.: wX(\hat{y}-y)\ge \Delta(\hat{y}-y) -\epsilon \forall y\neq \hat{y}
\end{eqnarray*}
where, $\Delta(\hat{y}-y)$ counts the number of elements in which $\hat{y}$ and y differ.
The number of constraints is $K^N-1$. We hope that the cutting plane methods\cite{joachims2009cutting} 
which start by removing all constraints and  iteratively keeps adding most violated constraints would work.
\section{Prediction}

Prediction is an integer programming problem(which is NP complete):


\begin{eqnarray*}
y^*=\arg \max _y wXy\\
st: y_{i,j}^{l,k}\le y_i^l\\
y_{i,j}^{l,k}\le y_j^k\\
\forall i \sum_{j=1}^{K} y_i^j = 1
\end{eqnarray*}
 


We hope we can prove that the LP-relaxation of this problem has provable approximation bounds.
\bibliographystyle{alpha}
\bibliography{mrf_formulation}
\end{document}
